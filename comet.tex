\documentclass[5p,authoryear]{elsarticle}
\usepackage{amssymb}
\usepackage{amsmath}
\usepackage{url}
\usepackage{caption}
\usepackage{subcaption}
\usepackage{graphicx}
\usepackage{color}

\journal{Astronomy and Computing}

\begin{document}

% Standardised display of inline code snippets.
\newcommand{\code}[1]{\texttt{#1}}

\begin{frontmatter}

\title{Comet: A VOEvent Broker}

% Trying to list here all the people who have directly contributed code or
% testing to TraP development and/or are expected to write text for this
% paper. Then the rest of the TKP in alphabetical order.
\author{John Swinbank}
\ead{j.swinbank@uva.nl}

\address{Astronomical Institute ``Anton Pannekoek'', University of Amsterdam, Postbus 94249, 1090 GE Amsterdam, The Netherlands}

\begin{abstract}

Enemy ships detected in sector 47!

\end{abstract}

\begin{keyword}
%% keywords here, in the form: keyword \sep keyword

%% MSC codes here, in the form: \MSC code \sep code
%% or \MSC[2008] code \sep code (2000 is the default)

\end{keyword}

\end{frontmatter}

\section{Introduction}
\label{sec:intro}

The exploration of the astrophysical time domain through timely follow-up
observations of transients offers the potential of many and varied potential
scientific results. However, to achieve these results it is necessary to
overcome a range of technical challenges in terms of identifying and
classifying transients, disseminating notifications of them to the community,
and coordinating follow-up efforts.

Mechanisms for distributing news of transient events already exist: both the
NASA Gamma-ray Coordinates Network\footnote{\url{http://gcn.gsfc.nasa.gov/}}
and The Astronomer's
Telegram\footnote{\url{http://www.astronomerstelegram.org/}} have long and
distinguished track records of enabling transient astronomy. However, the next
generation of large-scale survey telescopes such as Gaia, SKA and LSST promise
hundreds, thousands or even millions of transient detections every day. The
sheer volume of events presents a massive scalability challenge: it is no
longer practical for even large teams of astronomers to consider manually
reading, understanding and responding to these notifcations. Instead,
automation is essential. Furthermore, the diverse nature of these transient
hunting facilities---covering not only electromagnetic gamut from
low-frequency radio to space based X- and $\gamma$-ray monitors, but also
potentially including other types of signal such as gravitational
waves---means that a flexible and adaptable machine-readable mechanism must be
adopted for describing transients.

In an effort to address these challenges, the IVOA has introduced the
`VOEvent'\footnote{\url{http://www.voevent.org/}} \citep{Seaman:2011}
standard. VOEvent provides a standardized, machine- and human-readable way of
describing a wide range of transient astronomical phenomena. An individual
VOEvent describes a particular transient event, providing not only information
about what has been observed and how the observations were made, but also
making it possible for the author to include a scientific motivation for why
this particular event is interesting. Furthermore, a VOEvent may cite other
VOEvents, providing more information about a given transient or, if necessary,
superseding or retracting an earlier message.

VOEvents are published as XML \citep{Bray:2008} XML documents which should be
in compliance with schema \citep{Gau:2012, Peterson:2012} produced by the
IVOA. Working in XML enables VOEvent to make extensive use of other relevant
IVOA standards (?? cite STC, ucd, etc?) and enables convenient processing with
a wide range of standard commercial and open-source tools.

The VOEvent standard defines the structure and content of a VOEvent document,
but it does not describe a mechanism by which the author of a VOEvent may
distribute it to potentially interested recipients. This transport-agnosticism
is intentional: it is intended to provide the maximum possible flexibility, as
individual projects may disseminate events by whatever means best meets their
science goals. However, it is widely recognized that a baseline specification
for a simple transport protocol is of value in terms of providing a common
starting point for building international VOEvent distribution networks
\citep{Williams:2012}. The VOEvent Transport Protocol
\citep[VTP;][]{Allan:2009} is now seeing widespread adoption as such a
baseline.

This manuscript describes Comet, an implementation of all the components
necessary for interacting with the VOEvent Transport Protocol while acting as
a testbed for production-level VTP deployments and for new technologies and
``value-added'' services to assist in addressing the event deluge.  In
\S\ref{sec:vtp} a description of the protocol is provided to set the tool in
context. \S\ref{sec:design} describes how Comet has been designed and built to
meet the protocol specifications. \S\ref{sec:addedvalue} describes how Comet
builds upon VTP to help address future challenges in VOEvent filtering and
selection. \S\ref{sec:perf} considers the performance implications of
deploying VTP in support of next-generation astronomical infrastructure,
considering both the scalability of the protocol to large numbers of events
and to high latency connections. In \S\ref{sec:security} we consider the
security implications of VOEvents, how they can be mitigated at the transport
level, and describe a system being prototyped in Comet.  \S\ref{sec:avail}
describes the terms under which Comet is available and how to download,
install and use it. The results are summarized and conclusions drawn for
future of VOEvent transportation systems in \S\ref{sec:conclusions}.

\section{VOEvent Transport Protocol}
\label{sec:vtp}

\section{The Design and Implementation of Comet}
\label{sec:design}

\section{Added Value Services}
\label{sec:addedvalue}

\subsection{XPath-Based Filtering}
\label{sec:addedvalue:xpath}

\section{Performance}
\label{sec:perf}

\subsection{Total Throughput}
\label{sec:perf:total}

\subsection{High-Latency Connections}
\label{sec:perf:latency}

\section{Security}
\label{sec:security}

\section{Availability}
\label{sec:avail}

\section{Conclusions}
\label{sec:conclusions}

\section{Acknowledgements}
\label{sec:ack}

The author is grateful to Bob Denny and Alasdair Allan for many useful
discussions on the design and implementation of the VOEvent Transport
Protocol. I also thank Roy Williams and Tim Staley for their feedback on the
design and capabilities of Comet. This project was funded by European Research
Council Advanced Grant XXXXXX `AARTFAAC'.

\section*{References}

\bibliographystyle{elsarticle-harv}
\bibliography{comet}

\end{document}
